\chapter{总结与展望}

关于系统设计及实现至此就全部阐述完成了,在系统设计过程中遇到许多设计困难,在设计方面做了许多尝试、走了不少弯路。本章对整个毕业设计的过程、毕业设计的过程中所出现的各种问题做一个总结。另外,本章还对将来的一些研究做出一些展望。

产品设计的过程其实就是建模的过程,对产品业务的设计归纳为数据库模型,而用户交互设计则归纳为交互模型,在产品设计的过程当中

\section{MVVM~架构与~Reactive~UI}

由于接触过 WPF 框架下的 MVVM 模式,在开始毕业设计之初的目的就是在 JavaScript 的平台上实现一个 MVVM 框架并进行一次正式的 MVVM 开发实践,随着研究的深入,我对 MVVM 架构的理解也变得更加透彻起来,也认识到 Observer 模式在解决异步编程方面发挥着的重要作用。我个人对于小而美的架构设计有强烈的偏好,所以开始尝试设计一款轻量级的 MVVM 架构应用到项目中。

在函数式语言的领域内我找到了一个答案—— Functional Reactive Programming。后来在研究过程中我发现 Observer 与 Monad 的共同点,而 Monad 所采用的模式能够被 JavaScript 轻易地实现,于是就产生了 Model-View-ReactiveModel 模型(使用 Monad 替代 Observer 的 MVVM 实现)。

在项目中的这个实现命名取自量子纠缠(Entanglement),作为尝试性的实现,Entangle 虽然能够达到维护客户端数据的目的,但也还是存在着一些不足。

虽然已经将 entangle 的核心设计减小到了一定的程度,entangle 在 API 的定义上还没有经过梳理,易用性和代码可读性上都不能真正达到产品的标准~\cite{Gerken:2010:UCM:1753846.1754082}。这些不足体现在:

\begin{itemize}
  \item 建立 Monad 之间关系的过程较繁杂
  \item 与外部库对接上存在 API 接口一致性问题
  \item 受到 Builder 模式的限制,不同类型的转换子无法分类管理
  \item 内存管理模型不明确,存在一定风险
\end{itemize}

出于 API 的简洁性和易用性考虑,我将 entangle 设计为 Builder 模式的结构,Builder 模式隐藏了 Monad 之间关系建立的细节,但是随着项目的开发,Builder 模式的局限性也逐渐显示出来。首先,在 Builder 模式中无法实现对转换子命名空间,也就失去了对转换子进行分类管理的能力; 其次,Builder 模式受制于 JavaScript 语言平台在 API 接口上存在不公平,比如使用“点”操作定义串接 (flow) 关系而使用 $fork$ 函数定义并联关系; 另外,当前实现的 entangle 的语义并不完全,应对某些复杂的关系时难以使用(比如循环逻辑)。

JavaScript 拥有非常丰富的第三方库支持,对 DOM (Document Object Model) 的操作也离不开 jQuery 库的支持,现在的 entangle 仅能够通过 monkey patch 的方法将部分 jQuery 方法自动转换为 entangle 构造器方法,诸如事件注册类的 jQuery 方法只能通过编码实现,对函数式编程扩展库 underscore 和 lodash 也是相同的状况。

\section{研究展望}

接下来的开发工作希望在课程管理系统的实践基础上,将业务逻辑从课程管理延伸到通用 OA 应用领域、实现更灵活的控制配置。这些工作需要更复杂的需求和数据库结构进行支撑。

架构研究方面,为了克服上文提到的一些不足,考虑在 JavaScript 的基础上设计或选择一门编译至 JavaScript 代码并且能够与 JavaScript 执行环境进行互操作的函数式语言用于定义 ReactiveModel 以解决 JavaScript 语言平台的局限~\cite{Freeman:2012:HLW:2480361.2371413},使用更简单的形式定义 Monad 之间的关系。另一方面,接下来还将计划对内存控制模型进行分析和定义,增加内存保护措施,提高程序可靠性。

此外,基于 Monad 的 FRP 作为一种通用的异步交互处理方法除了其在系统前端开发中的应用之外理论上还可以应用在异步服务器交互的开发过程中。未来的研究也将更进一步探索 Monad 在图形界面开发之外的应用,尝试将 Reactive Programming 应用在面向事件的服务器平台开发中。

