\chapter{总结与展望}

关于系统设计及实现至此就全部阐述完成了,在系统设计过程中遇到许多设计困难,在设计方面做了许多尝试、走了不少弯路。本章对整个毕业设计的过程、毕业设计的过程中所出现的各种问题做一个总结。另外,本章还对将来的一些研究做出一些展望。

\section{MVVM~架构与~Reactive~UI}

由于接触过 WPF 框架下的 MVVM 模式,在开始毕业设计之初的目的就是在 JavaScript 的平台上实现一个 MVVM 框架并进行一次正式的 MVVM 开发实践,随着研究的深入,我对 MVVM 架构的理解也变得更加透彻起来,也认识到 Observer 模式在解决异步编程方面发挥着的重要作用。我个人对于小而美的架构设计有强烈的偏好,所以开始尝试设计一款轻量级的 MVVM 架构应用到项目中。

在函数式语言的领域内我找到了一个答案—— Functional Reactive Programming。后来在研究过程中我发现 Observer 与 Monad 的共同点,而 Monad 所采用的模式能够被 JavaScript 轻易地实现,于是就产生了 Model-View-ReactiveModel 模型(使用 Monad 替代 Observer 的 MVVM 实现)。

在项目中的这个实现命名取自量子纠缠(Entanglement),作为尝试性的实现,Entangle 虽然能够达到维护客户端数据的目的,但也还是存在着一些不足。

虽然已经将 entangle 的核心设计减小到了一定的程度,entangle 在 API 的定义上还没有经过梳理,易用性和代码可读性上都不能真正达到产品的标准~\cite{Gerken:2010:UCM:1753846.1754082}。这些不足体现在:

\begin{itemize}
  \item 建立 Monad 之间关系的过程较繁杂
  \item 与外部库对接上存在 API 接口一致性问题
  \item 受到 Builder 模式的限制,不同类型的转换子无法分类管理
  \item 内存管理模型不明确,存在一定风险
\end{itemize}

\section{研究展望}

接下来的开发工作希望在课程管理系统的实践基础上,将业务逻辑从课程管理延伸到通用 OA 应用领域、实现更灵活的控制配置。这些工作需要更复杂的需求和数据库结构进行支撑。

架构研究方面,为了克服上文提到的一些不足,考虑在 JavaScript 的基础上设计或选择一门编译至 JavaScript 代码并且能够与 JavaScript 执行环境进行互操作的函数式语言用于定义 ReactiveModel 以解决 JavaScript 语言平台的局限~\cite{Freeman:2012:HLW:2480361.2371413},使用更简单的形式定义 Monad 之间的关系。另一方面,接下来还将计划对内存控制模型进行分析和定义,增加内存保护措施,提高程序可靠性。

未来的研究还将更进一步探索 Monad 在图形界面开发之外的应用,尝试将 Reactive Programming 应用在面向事件的服务器平台开发中。

