\begin{center}
\zihao{3}\textbf{An MVVM Based Course Management System}
\end{center}
\begin{center}
\zihao{-3}\textbf{ABSTRACT}
\end{center}
\vspace{2mm}
\zihao{-4}

Model-View-ViewModel (MVVM) design pattern is a pattern providing a resolution to asynchronous programming of Graphical User Interface (GUI) which is introduced into Windows Presentation Framework (WPF) at 2005 by Microsoft. Based on Observer Pattern, MVVM uses feature called DataBinding to make connection between Model, ViewModel and View objects. With it's success in handling asynchronous events, MVVM has been adopted widely Rich Interaction Applications (RIAs).

The subject of this article is to design a Course Management system based on MVVM pattern.

Functional Reactive Programming is an idea from Haskell designed to implement reactive animation at first. In contrast to imperative programming method, which handles the dependency between data based on order of statements, declaritive language defines dependency delicately at the time a program is designed. Programs composed in imperative programming language could be very sensitive to the implementation of asynchronous logic for a heavy burden of reducing unnecessary dependencies.

A typical Object-oriented MVVM implementation on JavaScript platform does not make use of functional programming ability of JavaScript. The article tries to introduce the idea of Functional Reactive Programming (FRP) into MVVM design and analyse the use of Observer in typical MVVM implementation from the view of Functional Programming Language. Then, bring up the ViewModel built upon Monad objects.

\vspace{3mm}
\zihao{-4}\textbf{KEY WORDS}\quad MVVM \quad RIA Development \quad Reactive Programming
