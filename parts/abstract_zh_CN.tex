\newpage
\begin{center}
	\heiti\zihao{3}\textbf{基于 MVVM 架构的课程管理系统的设计与实现}
\end{center}
\begin{center}
	\heiti\zihao{-3}\textbf{摘\quad 要}
\end{center}
\vspace{2.5mm}
\songti\zihao{-4}

Model-View-ViewModel (MVVM) 模式是 Model-View-Controller (MVC) 模式的一种特化,2005 年由 Microsoft 提出并首次应用于 Windows Presentation Framework (WPF),MVVM 模式将 MVC 架构模式与 Observer 模式配合使用,在 Model、ViewModel 与 View 之间通过 DataBinding 建立关系,解决了图形用户界面(Graphical User Interface)开发中对异步事件处理的难点,近年来十分受到业界的重视。

本文以课程管理系统的设计与实现为线索,对比分析 MVVM 模式的开发性能,同时尝试对 MVVM 模式在 JavaScript 语言平台上的实现提出改进方案。借助 JavaScript 语言平台支持函数式编程的特性,在 MVVM 框架中引入 Functional Reactive Programming (FRP) 的设计思想,并从 Reactive Programming 的角度分析讨论 MVVM 模式在处理异步逻辑方面的优越性、设计基于 Monad 对象实现 ViewModel 的改进框架。

\vspace{3mm}
\zihao{-4}\heiti\textbf{关键字}\quad \songti MVVM \quad 富应用 \quad Reactive Programming \quad 课程管理系统

