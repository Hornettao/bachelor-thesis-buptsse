\newpage
\begin{center}
	\heiti\zihao{3}\textbf{基于 MVVM 架构的课程管理系统的设计与实现}
\end{center}
\begin{center}
	\heiti\zihao{-3}\textbf{摘\quad 要}
\end{center}
\vspace{2.5mm}
\songti\zihao{-4}

Model-View-ViewModel (MVVM) 模式是 Model-View-Controller (MVC) 模式的一种特化,2005 年由 Microsoft 提出并首次应用于 Windows Presentation Framework (WPF),MVVM 模式利用 Observer 模式,在 Model、ViewModel 与 View 之间通过 DataBinding 建立关系。MVVM 模式的出现解决了图形用户界面(Graphical User Interface)开发中对异步事件处理的难点,近年来十分受到业界的重视。

本项目通过设计与实现基于 MVVM 设计模式的课程管理系统,在验证 MVVM 模式性能的基础上对 MVVM 模式在 JavaScript 语言平台上的实现提出改进方案。

JavaScript 语言平台支持函数式编程的特性,在 MVVM 框架中引入 Functional Reactive Programming (FRP) 的设计思想,本文将使用 FRP 的角度分析讨论 Observer 模式在 MVVM 模式中的应用,并设计基于 Monad 对象实现 ViewModel 的改进框架。

\vspace{3mm}
\zihao{-4}\heiti\textbf{关键字}\quad \songti MVVM \quad FRP \quad 课程管理系统 \quad 富应用

