\chapter*{外\quad 文\quad 译\quad 文}
\pagestyle{empty}

\begin{center}
{\heiti\zihao{3}Brain Computer Interface System: Progress and Prospects

\textsl{By Brendan Z Allison, Elizabeth Winter Wolpaw and Jonathan R Wolpaw}

\heiti 脑机接口系统:进展与展望
}
\end{center}

\zihao{-4}

\section*{摘要}
脑-机接口系统支持直接神经活动测量的通信而不需要任何肌肉的活动。它可以为严重的神经肌肉失调患者提供最好有时甚至是唯一的交流方式,也可以最终能为稍微不太严重的残疾人或者是正常人在很广的范围内提供应用。这篇综述讨论了脑-机接口系统的结构和功能,定义了其中的术语,阐明了实际的应用。该领域目前的进展和未来的机会也被明确。


成百万(Millions)的人患有严重的神经肌肉功能失调症,比如神经渐冻症(ALS),大脑中风,小儿麻痹症,肌营养不良,多发性硬化和巴士综合症等。这些病的很多患者无法通过通常的神经肌肉通道控制进行交流而必须依靠于其他使用其残存的肌肉控制能力的设备进行交流,比如眨眼,肌电图(EMG),呼吸。典型的,这些人使用这样的方式交流是因为他们无法使用传统的交互方式,比如键盘,鼠标和其他的需要很大肌肉控制能力的交流方式。
	
脑-机接口系统是一个能让人不需要使用外围神经或者肌肉就发送信息或者命令的通信系统。它从大脑记录信号并将其转换成有用的交流信息。因此,脑-接接口系统对于那些没有有效肌肉控制能力的人来说是有用的。这篇综述描述了一个脑-机接口系统的基本组成和当前脑-机接口的主要分类,定义了脑-机接口文献中的一些术语并考察了未来几年该领域的发展趋势。

\section*{综述(Overview)}
脑-机接口系统可以使用的信号来自四个地方,如FIGURE 1所示:与身体不接触的传感器,比如功能核磁共振(fMRI) 或者脑磁波扫描(MEG);从头皮表面通过脑电图EEG电极或者功能近红外光谱;从硬膜表面或大脑表面使用脑皮层电极(ECoG);或者使用嵌入到大脑皮层的微电极从大脑内记录到的信号。从这些区域采集到的信号,不管是来自于健康人还是严重残疾的病人,都可以被记录而转变为通信和控制的方式。

FIGURE 1 展示了任何脑-机接口系统的组成。一个脑-机接口系统有四个必须组成部分:从以上描述的区域记录脑电信号的信号采集部件,有软件能提取脑电信号特征用于脑-机接口应用并有转换算法能这些提取的特征转化成设备指令的信号处理部件,实现命令的输出设备以及管理这些组件如何交互的控制协议。

\section*{脑-机接口信号采集}
大多数脑-机接口应用不需要手术来植入电极,因此而被成为无创脑-机接口或非侵入式脑-机接口。当前,几乎所有的该类脑机接口系统使用安放在头皮表明的脑电图(EEG)传感器来记录大脑活动。而这也是本篇综述主要关注的一类脑-机接口系统。通过手术植入电极到大脑皮层表面或内部或其他大脑区域来采集信号的脑-机接口系统被认为是侵入式的。基于脑皮层ECoG记录的脑-机接口系统是侵入式的,因为这类脑-机接口系统需要手术,但是也比皮质内的脑机接口系统的侵入式程度低,因为脑皮层ECoG电极并不会进入大脑而只是分布在大脑表面。

侵入式电极可以比非侵入式系统描述更加详细的大脑活动情况。因为头皮的存在弄脏,削弱和过滤了大脑电活动强度,所以侵入式电极比头皮表面的电极有更好的空间精度,更强的信号强度和更广的信号频率范围。比如,脑皮层ECoG记录的脑-机接口系统能记录到与运动相关100-200Hz的大脑活动,这个频率范围远远大于头皮脑电所能记录到的频率范围。此外,侵入式脑机接口可以24小时使用,需要很少的准备时间和清理时间,对于肌肉电噪声和外界噪声更不敏感。然而,当前的侵入式脑-机接口系统提供几乎和非侵入式系统近似的性能和表现。更进一步,侵入式脑-机接口伴随着高昂的手术费用,伤疤,感染的危险以及常规的医学检查而且其长时间的稳定性仍然不清楚。因此,侵入式脑-机接口值得更进一步的研究,而当前大部分的病人和研究人员可以从容的选择非侵入式的方案。

\section*{基于脑电图(EEG)的脑-机接口系统}

在进行运动或者想象运动时从感觉运动皮层记录到的u(8-12Hz)和beta(12-30Hz)EEG脑电会减弱。这种在活动之前的同步减弱被称为时间相关去同步(Event-related desynchronization)。Wolpaw和他的同事介绍的机遇想象运动的脑-机接口系统证明不管是残疾人还是正常个体都可以通过学习使用mu和beta波的节律在一维空间上控制鼠标。在那之后不久,Wolpaw和McFarland证明了mu和beta节律可以被用来进行二维控制。为了使大多数使用者能掌握使用mu和beta节律的幅度来控制一维的运动而且能使系统最优化适应使用者的节律,需要五到十个时长为24分钟的训练阶段。典型的,使用者首先练习如何控制鼠标以恒定速率从上往下垂直移动。通过额外的训练,使用者可以学会使用两个大脑半球不同的mu和beta节律来获得独立两个控制通道因而最终达到精确的二维控制。第三个通道或者可以被用来做类似鼠标点击的选择。在训练开始阶段,使用者通常通过想象来调制mu和beta节律但是当他们变的擅长于此的时候,任务显得更加自动化而他们也不再需要想象来控制。在以上这些研究中,身体正常的使用者和残疾的使用者都学会了控制mu和beta节律在一维或者二维移动计算机鼠标,完成简单的字处理任务和在计算机屏幕上选择选项。有些实验被试能通过最小的训练通过使用mu和beta节律相关的想象运动完成简单的控制。基于mu节律的脑-机接口系统也被用来控制设备,比如矫正器或假肢等。

Birbaumer和他的同事开发了基于慢皮层电位(Slow Cortical Potentials)的脑-机接口系统。慢皮层电位主要是和感情以及想象相关的低频EEG电压变化相关。一个典型的基于慢皮层电位的脑-机接口系统在一个2s的休息周期内测量EEG脑电,然后再一个2s的活跃活跃阶段内使用者会产生一个或正或负的慢皮层电位。基于慢皮层电位的方法需要1到5个月的训练。身体不残疾和残疾人,包括重度神经渐冻症患者都可以使用慢皮层电位进行字处理操作和完成其他任务。然而,基于慢皮层电位的通信是肯定很慢的淫威探测到慢皮层电位的改变需要好几秒的时间。

虽然基于慢皮层电位和基于感觉运动(mu和beta节律)的脑-机接口系统通常需要一些用户训练,但是其他的脑-机接口系统可以不依赖于操作条件而只需要很少的训练。比如,当刺激被识别和分辨以后就通常能激发的属于事件相关电位的P300成分就能被用来作为脑-机接口系统。Farwell和Donchin开发了基于P300的脑-机接口系统。该系统呈现给使用一个包含字母和其他元素的矩阵在计算机屏幕上。矩阵中单独的行或者列以随机的顺序飞快的闪烁,使用者被告知默数他/她想交流的那个符号闪烁的次数而同时忽略其他的闪烁。包含目标的行或者列会激发P300反应,因此大多数人都是可以使用该脑-机接口系统的。虽然最初担心P300信号可能在神经渐冻症病人身上很微弱或者P300信号强度在持续使用之后会减弱,但是最初的数据还是表明基于P300的脑机接口系统可以有效的被神经渐冻症病人使用(每天几个小时持续几个月)。基于P300的脑-机接口系统也被发展成基于使用者的听觉刺激,特别是对那些失明患者。最近的工作表明基于P300的脑机接口系统在大多数被试身上可以提供比基于慢皮层电位(SCP)和感觉运动(sensorimotor)的脑机接口系统提供更好的性能。此外不同脑-机接口方法在各种不同被试身上的效果的研究还比较空白。

稳态视觉诱发电位翻译了对于快速震荡刺激信号的注意力。如果使用者直接关注一个这样的刺激,枕骨区域与频率相关的大脑活动可以被用来推测使用者的目的。脑机接口系统也可以方便的使用其他的稳态现象,比如被不同震动频率所诱发的稳态触觉诱发电位。同时对两个或更多个刺激的同时注意能力,比如两个视觉刺激或者不同调制的两个刺激,或许可以用来改善信息容量。

EEG脑电频谱会因为用户执行一些一般的心理活动而变化,比如写陈述,进行算术运算。有些脑-机接口系统通过决定一个用户执行不同的任务来完成通信。这种方式和其他大多数脑-机接口系统类似的都只能提供很低的信息容量因为通过EEG脑电来辨认不同的大脑任务活动和在不同任务间进行转换都很慢。

\textit{[跳过与EEG脑电不相关的三段:ECoG-based BCIs, Intracortical BCIs and BCIs using nonelectrical signals]}

\section*{脑机接口系统信号处理}
一旦信号被上面所提及的方法采集到,脑机接口系统的第二部分,信号处理单元(FIGURE 1)就提取信号特征并将其转换成信息或者命令。信号处理单元首先分析原始信号然分立脑机接口系统能使用的特征。特征提取通常都使用相对简单的方法,比如自回归信号分析(Autoregressive Frequency Analysis)或者更复杂的技术,比如独立分量分析(Independent component analysis, ICA)。转换算法然后将这些特征翻译成输出命令。为了让一个脑-机接口系统应用于实际的通信,该系统需要快速和在线的执行所有这些任务。转换算法根据复杂度种类很多,从线性方法比如分段判别式分析或者加权频率求和到非线性的神经元网络自适应算法。

信号处理算法的参数需要适应单个使用者。理想情况下,一个脑-机接口系统为每个用户判断识别器最适宜的信号特征然后继续对这些特征中经常自发出现的量进行适应。

\section*{脑机接口输出设备}
在大脑信号的特征被提取和转换以后,脑机接口系统的第三个组成部分,输出设备将实现转换算法表达的信息或者命令。就当下而言,最常用的脑-机接口输出设备是电脑显示器。基于显示器的脑-机接口系统已经发展成能使用户通过移动鼠标做二选一,多选一,从下拉菜单选择选项,上网或者在虚拟现实中行走。有些脑-机接口系统使用扬声器或者耳机提供听觉刺激或者反馈信号。脑-机接口也被用来控制开关,日常家电,比如空调,电视或音乐播放器,媒体设备,机器手,可移动机器人,功能性电刺激器或矫正器和一个全向的飞行模拟器。

\section*{脑机接口操作协议}
操作协议定义了使用者大脑与脑机接口系统之间实时的交互。它提供了一个使用前端和平台来管理其他三个模块与操作系统之间如何相互通信,协调用户系统交互的具体细节,那个选项是用户可用的,何时用户的活动将以何种方式影响控制以及反馈的本质和时序。

类似于其他的脑-机接口组件,操作协议在过去几年取得了巨大的进步。很多论文讨论了时序,反馈和易用性。基于EEG脑电活动的错误纠正,比如错误相关负波,P300或其他测量可以提高脑机接口系统对某些用户的使用性能。现在的有些操作协议比早起的大多数脑-机接口系统提供更多的,这些操作协议要么提供更多的选项,或者让用户在不同的面板选项中选择优势也通过使用菜单。

最广泛使用的脑机接口操作系统是BCI2000。目前他正被全世界超过150家实验室使用。BCI2000提供的许多特性使其对研究人员非常有吸引力。它对于很多的信号采集,信号处理和输出系统都提供了相当高的灵活性和内部整合能力。该系统也提供了各种各样的实时与线下的分析手段,并且是免费供研究使用的。

\textit[此处跳过五个小节:Devices that are not BCIs, Other terminology, BCI versus BMI, Dependent versus independent BCIs]

\section*{同步脑机接口系统 V.S 异步脑机接口系统}
在同步脑机接口系统里,操作的时序是有脑机接口系统决定的,而不是用户。比如,大多数的P300脑-机接口系统让用户观察的闪烁信号速率都是由系统控制的。大不多的基于mu节律和慢皮层电位的脑机接口系统也是在BCI系统要求用户产生大脑活动信号时操作才能进行。相比较而言,在异步的脑机接口系统里是用户在控制通信的节奏。比如,某些脑机接口系统允许用户自主的进行想象运动或者以他们自己的节奏进行某种心理活动。但是异步脑机接口系统或许对于可以被理解为信息或者命令的不相关大脑活动更敏感。这个问题已经预料到的在探索的解决办法是通过大脑活动来关闭或者打开脑机接口系统。

\textit{[此处略去两节:Automaticity, Terminology relating to the degree of a patient’s disability]}

\section*{把脑机接口系统从实验室搬回家}
脑机接口研究的最终目的是开发出能为残疾人提供沟通交流,方便日常生活和改善其生活质量的脑机接口系统。虽然很多研究组都报告了脑机接口系统在重度残疾人身上的使用,包括一些在家庭环境下的使用,但是脑机接口系统还无法满足大部分残疾人日常生活中最重要的需求。某些工作还刚刚开始。

一些关键性因素阻碍脑机接口系统被广泛的采纳。很少有健康专家和恢复专家知道脑机接口系统能为他们的病人提供什么。虽然健康专家和家庭成员都认为患有慢行神经运动能力减弱的人很悲观而不像继续活下去,但是研究表明这些患病的个体一点也不必健康的正常人更悲观。如果能提供好的支持系统和一种通信方式,他们能过他们认为愉快而有意义的生活。

长期在病人家中使用脑机接口系统需要病人家属和护工的持续支持,也包括某些最基本的技术支持。脑机接口系统研究和发展面临的最大障碍是使系统变的更加用户友好和健壮以使护工和患者能在日常中不需要太多困难就能操作他们。

保险赔付是另外一个重要的议题。当前,医疗陪护指导原则允许的通信保险补偿对象必须提供速度。因此在当前的指导原则下,只能提供字处理,网络浏览,设备控制和环境控制能力的脑机接口系统并没有被包括在内。然而,音频输出,包括像语音输出可以被毫不费力的加到目前的大多数系统上。很多病人可能并没有医疗保障。除此之外,由于持续的对脑机接口系统的技术支持可能比购买脑机接口系统本身更加昂贵,因此开发易于使用,稳定和能自适应的满足用户和护工需要的脑机接口系统非常的有必要。

严重残疾病人的喜好有时很难被预测。比如,Kubler和他的同时秒速了一个已经好几个月不能交流的病人。她通过脑机接口系统提出的第一个要求是有关她的衣服和修指甲。和一个正常人想象生活无法移动和交流时的情况比起来,这很惊人。同时这也强调了进一步研究的需要和付出努力使脑机接口系统更符合病人的能力,需要和渴望。

当前,并没有商业个体对普通的病人群体提供临床实用的脑机接口系统。因此,当前的脑机接口通常都是针对研究脑机接口系统和协议的某些特定研究团队的成员,而不是满足病人需要和能力的系统。由于有些人比其他人在使用某种特定脑机接口系统方法时有更好的表现而导致了不同的喜好和目标,因此确保潜在的用户都有一定脑机接口选择是一件重要的事。

虽然脑机接口系统优势以其信息容量和数据传输速率作为衡量和评价他们的标准,但是必须意识到某些不同于速率和精确度的因素也很重要。比如,某些用户跟喜欢慢速的通信平台因为这样的系统更容易使用,更不容易让人疲劳,更容易定制,更稳定而且也更合适。可用性,侵入性,价格,便携性,训练时间,在线系统和个人支持,外观以及使用系统需要的时间和技术也同时是重要的考虑因素。

\newpage
\section*{外文译文原文}
\begin{center}
\includegraphics[width=\textwidth]{graphic/PaperReview/Review1.PNG}
\end{center}

\begin{center}
\includegraphics[width=\textwidth]{graphic/PaperReview/Review2.PNG}
\end{center}

\begin{center}
\includegraphics[width=\textwidth]{graphic/PaperReview/Review3.PNG}
\end{center}

\begin{center}
\includegraphics[width=\textwidth]{graphic/PaperReview/Review4.PNG}
\end{center}

\begin{center}
\includegraphics[width=\textwidth]{graphic/PaperReview/Review5.PNG}
\end{center}

\begin{center}
\includegraphics[width=\textwidth]{graphic/PaperReview/Review6.PNG}
\end{center}

\begin{center}
\includegraphics[width=\textwidth]{graphic/PaperReview/Review7.PNG}
\end{center}

\begin{center}
\includegraphics[width=\textwidth]{graphic/PaperReview/Review8.PNG}
\end{center}

\begin{center}
\includegraphics[width=\textwidth]{graphic/PaperReview/Review9.PNG}
\end{center}

\begin{center}
\includegraphics[width=\textwidth]{graphic/PaperReview/Review10.PNG}
\end{center}

\begin{center}
\includegraphics[width=\textwidth]{graphic/PaperReview/Review11.PNG}
\end{center}

\begin{center}
\includegraphics[width=\textwidth]{graphic/PaperReview/Review12.PNG}
\end{center}
