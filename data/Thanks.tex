\chapter*{致\qquad 谢}
\songti\zihao{-4} 

	本毕设的完成得到了很多人的帮助。首先要感谢清华大学医学院神经工程实验室的洪波老师给了我这次机会在他指导下完成毕设。实验室的高小榕老师在毕设期间也给与了我很多的指导,子钟同步的方案就是在他的提议才完成的。最后还要感谢神经工程实验室的Director高上凯教授,清华大学医学院的王广志教授以及本人本科所在学校北京邮电大学,我能跨校跨专业顺利完成本毕设的任务,离不开学校政策和上级老师的支持。

	此外,在本毕设的完成过程中,我还得到了清华大学医学院神经工程实验室很多学长和学姐的帮助。感谢胥红来学长在我整个毕设过程中对于技术方面和思考问题方式方面的指导和帮助;感谢张丹学长从一开始就给予我的热情支持和鼓励;感谢高海洋师兄对我论文格式排版等方面的指点;感谢钱天翼学长在Matlab处理数据上的指导以及在最后毕设答辩PPT中提出的建议;感谢肖小军工程师帮我完成了QFN封装芯片的焊接以及其他技术方面的支持;感谢宾广宇学长,黄肖山学长还有刘涛学长在计算机端数据处理程序方面的支持,另外刘涛学长提供给我的一些文献帮助我很快入门。感谢郭靖学姐在文献阅读方面的指导,她慷慨的提供了硕士论文原稿给我做为参考;

	本毕设所完成的无线同步触发器协议设计与实现中还有不少问题,由于时间的原因并没有给与完善,衷心希望我所做的工作能为未来该系统的完善提供一些方便。


